% LaTeX resume using res.cls
\documentclass{res}
%\usepackage{helvetica} % uses helvetica postscript font (download helvetica.sty)
%\usepackage{newcent}   % uses new century schoolbook postscript font  
\topmargin=-0.5in %length of margin at the top of the page (1 inch added by default)
\headheight=-1pt %1in margins at top and bottom (1 inch is added to this value by default)
\setlength{\textheight}{10.25in}

\addtolength{\oddsidemargin}{-.475in}
\addtolength{\evensidemargin}{-.475in}
\addtolength{\textwidth}{0.75in}

\begin{document} 

\begin{center}
{\Large \textbf{Benjamin Shih}}
\end{center}
\vspace{-0.1in}
benjshih@gmail.com \hfill United States Citizen
\\*http://benshih.github.io \hfill https://github.com/benshih


\vspace{-0.1in}
\section{Education}
\vspace{0.05in} 
 
University of California, San Diego (UCSD) \hfill San Diego, CA
\\* Ph.D. Mechanical and Aerospace Engineering \hfill August 2015 - present
 
Carnegie Mellon University \hfill Pittsburgh, PA
\\*M.S. Electrical and Computer Engineering \hfill August 2013 - December 2013
\\*B.S. Electrical and Computer Engineering \hfill August 2009 - May 2013

Montgomery Blair High School \hfill Silver Spring, MD
\\*Math, Science, and Computer Science Program \hfill August 2005 - June 2009


\vspace{-0.1in}
\section{Skills} % This is more relevant for industry job applications.
\vspace{0.05in}
{\bf Software: } MATLAB, Eagle, SolidWorks, LaTeX, Git, Cadence, ProTools
\\*{\bf Electronics: } soldering, oscilloscope, function generator, multimeter, circuit simulation, PCB board design, microcontroller programming
\\*{\bf Coding: } C++, Python, Java, C, HTML
\\*{\bf Languages: } English (proficient), Mandarin Chinese (speaking), Spanish (basic)




\vspace{-0.1in}
\section{Experiences}
\vspace{0.05in} 

 {\bf Bioinspired Robotics and Design Lab, UCSD} \hfill San Diego, California
    \\*\emph{Graduate Research Assistant} \hfill September 2015 - present
\begin{itemize}
    \item Soft robotics.
    \item Advised by: Prof. Michael Tolley
\end{itemize}

 {\bf Momentum Machines} \hfill San Francisco, California
    \\*\emph{Embedded Software Engineering Intern} \hfill May 2015 - August 2015
\begin{itemize}
    \item Food technology startup using robotics and automation to produce gourmet food. 
    \item Requirements gathering and electronics interfacing of dozens of sensors and actuators. Used an ARM-based microcontroller.
    \item Lead engineer for PCB fabrication of 6 unique boards with a design firm.
    \item Statecharts (finite state machine) software architecture for embedded control. Used a web-based graphical user interface to facilitate rapid prototyping and fast system bringup.
    \item Prototyped and tested various subsystem mechanisms for mechanical engineering team.
    \item Advised by: Jeff Jensen, Ali Rathore.
\end{itemize}

 {\bf Reconfigurable Robotics Lab, EPFL} \hfill Lausanne, Switzerland
    \\*\emph{Research Assistant, $\acute{E}cole$ Polytechnique $F\acute{e}d\acute{e}rale$ de Lausanne} \hfill May 2014 - April 2015
\begin{itemize}
    \item Built untethered, locomotive robot using soft pneumatic actuators (SPAs).
    \item Experimented with actuator frames to improve actuation consistency.
    \item Automated SPA testing using computer vision.
    \item Advised by: Prof. Jamie Paik, Dr. Juan Manuel Florez.
\end{itemize}

 {\bf MIT Lincoln Laboratory} \hfill Lexington, MA
    \\*\emph{Graduate Intern} \hfill May 2013 - August 2013
\begin{itemize}
    \item Worked with mechanical engineering intern to equip plane with visible spectrum vision capabilities. 
    \item Created user interface using Qt for streaming video from camera and toggling individual frame recording.
    \item Designed software architecture using UML diagrams to describe how camera interacts with system.
    \item Team: Adith Subramanian. Advised by: Dr. Jon Watson, Dr. Seth Trotz, Dr. Jim Truitt.
\end{itemize}

    {\bf NanoJapan, Rice University} \hfill Houston, TX
    \\*\emph{Undergraduate Researcher} \hfill May 2011 - August 2011
\begin{itemize}
    \item Analyzed vibrational and rotational modes of C$_{60}$ nanocars via Raman spectroscopy.
    \item Delivered poster presentation at International Symposium on Terahertz Nanoscience (TeraNano) at Osaka University, Japan in November 2011.
    \item Worked in cross-cultural research setting alongside $\sim$40 Japanese graduate students.
    \item Advised by: Prof. Kevin Kelly.
\end{itemize}

  




\section{Projects}

{\bf Dekoboko} \hfill Berlin, Germany
\\*\emph{Intel IoT Hackathon 2015 \hfill April 2015}
	\begin{itemize}
		\item Combined GPS and accelerometer sensor for mapping vibration+location data while riding bike. LED bar sensor for displaying vibration intensity. Data logged to internet-connected Intel Edison board, pulled using Javascript, and displayed on Heroku page using Ruby on Rails.
		\item Award: First prize, \$1500 euro. 
		\item Semifinalist, Hackaday Prize 2015. Featured on Hackaday blog. Accepted for demo at Maker Faire San Diego 2015.
		\item Team: Daniel Rojas, Maxim Lapis.
	\end{itemize}
	
	
{\bf Your Augmented Reality Notes} \hfill Zurich, Switzerland
\\*\emph{HackZurich 2014 \hfill October 2014}
	\begin{itemize}
		\item Augmented reality system using Leap Motion and Oculus Rift for viewing Evernote notes. Also with android and iOS mobile app. Incomplete project integration.
		\item Award: Finalist (top 25 of 101). % percent \% escape char.
		\item Team: Brian Zhenbang Li, Diego Alfonso Ballesteros, Cifong Kang.
	\end{itemize}

    {\bf Real-Time Mosaicking and Tracking on UAV}  \hfill September 2013 - April 2014
\begin{itemize}
    \item Implemented computer vision algorithms using Node.js, OpenCV, and MATLAB for use on Parrot ARDrone.
    \item Conducted market research to identify clients. Met with companies and startup incubators to refine idea. 
	\item Team: Rentaro Matsukata
\end{itemize} 

    {\bf Roof-Shingling Robot}  \hfill January 2013 - May 2013
\begin{itemize}
    \item Managed four person team to design and construct roof-shinging robot from scratch, budget of \$1000. 
    \item Designed robot base, dropping, feeding, and dispensing mechanisms in SolidWorks CAD. PID controller programmed and tuned using Arduino. 3rd place in class competition, \$200 prize.
	\item Team: Ram Muthiam, Mark Erazo, Hao Wang. Advised by: Prof. John Dolan.
\end{itemize}

 {\bf Human-Computer Interaction Institute, Carnegie Mellon University} \hfill Pittsburgh, PA
    \\*\emph{Undergraduate Researcher} \hfill August 2012 - December 2012
\begin{itemize}
    \item Designed wearable wristband using vibrational microphones, which sent the on-body signals through an amplifier and to a computer as a stereo audio signal.
    \item Classified the training data responses obtained from the microphones by recording select locations on the arm and generating spectral features, including spectral density, spectral centroid, and band energy ratio, based on individual and cross-correlation of the audio channels using Weka. Obtained $\sim$95\% accuracy.
	\item Team: Murium Iqbal, Robert Xiao. Advised by: Chris Harrison, Prof. Bhiksha Raj.
\end{itemize}

    {\bf Line-Following Mobile Robot} \hfill October 2011 - April 2012
\begin{itemize}
    \item Worked with peer to create simple scheduler for pulsing motors and reading sensors.
    \item Handmade components: plexiglass chassis, two-link joint for front wheel steering, wheel encoders using black/white tape and infrared sensors, H-bridge for motor control, infrared sensor array for line detection. 
    \item Programmed PIC18F25K22 using C/assembly in MPLabX for controlling steering and monitoring sensors. 
    \item Team: Rentaro Matsukata.
\end{itemize}








  
\section{Honors}
Semifinalist, Hackaday Prize 2015 \hfill August 2015
\\* UCSD Departmental Fellowship \hfill February 2015
\\* Winner, Intel Internet of Things Hackathon, Berlin (1500 euros) \hfill April 2015
\\* Finalist (top 25 out of 101 projects), HackZurich Hackathon \hfill October 2014
\\*Honorable Mention, National Science Foundation (NSF) Graduate Research Fellowship Program \hfill April 2014
\\* Scholarship of Excellence in Research at EPFL (20k CHF) \hfill February 2014
\\*Small Undergraduate Research Grant, Carnegie Mellon University (500 USD) \hfill November 2011
\\*NanoJapan NSF International Research Experience for Undergraduates Program \hfill February 2011
\\*Intel Science Talent Search, Semifinalist (1000 USD) \hfill January 2009




\vspace{-0.1in}
\section{Conference Publications}
\vspace{0.05in}
% include acceptance rates.
% {\bf B. Shih}, J. M. Florez, J. Paik. "SiliBot: An Untethered Reconfigurable Soft Robot For Multi-Legged Locomotion". In preparation.


J. M. Florez, {\bf B. Shih}, Y. Bai, J. Paik. "Soft Pneumatic Actuators for Legged Locomotion". IEEE International Conference on Robotics and Biomimetics (ROBIO 2014), Bali, Indonesia. December 2014. Acceptance rate: 58.6\% (374 of 638).





\vspace{-0.1in}
\section{Academic Presentations, Lectures, Talks}
\vspace{0.05in}

{\bf Swiss National Centres of Competence in Research (NCCR) Annual Review} \hfill Nov 2014
	\begin{itemize}
    		\item Presented EPFL SPA crawler robot, along with RRL's soft rat exoskeleton, to Swiss funding agencies.
	\end{itemize}

    {\bf Carnegie Mellon CIT Dean's Council Presentation} \hfill November 2012
    \begin{itemize}
    \item Presented Hand Input project to the dean's council.
    \item Received positive feedback about project concept from audience. 
    \end{itemize}

    {\bf International Symposium on Terahertz Nanoscience} \hfill November 22, 2011
    \begin{itemize}
    \item Presented poster: Temperature-Dependent Raman Spectroscopy of Fullerene Nanocar Wheels by Benjamin Shih, Chad Byers, Albert Chang, Corey Slavonic, Dr. Kevin Kelly.
    \item Conference held at Osaka University in Osaka, Japan.
    \end{itemize}

    {\bf 25th Annual Rice University Quantum Institute Summer Colloquium} \hfill August 2011
    \begin{itemize}
    \item Presented poster: Temperature-Dependent Raman Spectroscopy of Fullerene Nanocar Wheels by Benjamin Shih, Chad Byers, Albert Chang, Corey Slavonic, Dr. Kevin Kelly.
    \item Conference held at Osaka University in Osaka, Japan.
    \end{itemize}


\vspace{-0.1in}
\section{Academic Mentoring}
\vspace{0.05in}

Basile Audergon. EPFL B.S. Soft Pneumatic Actuator Frame Fabrication. Dec 2014 - Apr 2015.
\\Nicolas Besuchet. EPFL B.S. Soft Pneumatic Actuator Frame Fabrication. Jan 2015 - Apr 2015.


\section{Teaching}
\emph{\bf Graduate}
\\ {\bf Electrical and Computer Engineering Department, Carnegie Mellon University} \hfill Pittsburgh, PA
    \emph{18-202 Mathematical Foundations of Electrical Engineering Teaching Assistant} \hfill August 2013 - December 2013
\begin{itemize}
    \item Weekly office hours to review math topics.
    \item Course by: Prof. Tom Sullivan.
\end{itemize}     

\emph{\bf Undergraduate}
\\ {\bf Electrical and Computer Engineering Department, Carnegie Mellon University} \hfill Pittsburgh, PA
    \emph{18-320 Microelectronic Circuits Teaching Assistant} \hfill August 2012 - December 2012
\begin{itemize}
    \item Guide $\sim$30 students through amplifier design (analog) and transistor layouts in Cadence (digital). Lead two 3 hour/week lab sections. 
    \item Course by: Prof. Jeyanandh Paramesh.
\end{itemize}     

 {\bf Electrical and Computer Engineering Department, Carnegie Mellon University} \hfill Pittsburgh, PA
    \emph{18-290 Signals and Systems Teaching Assistant}\hfill August 2011 - December 2011
\begin{itemize}
    \item Guided $\sim$30 students through various MATLAB activities related to introductory signal processing, including audio/speech processing and specgram analysis. Managed one 3 hour/week lab section. 
    \item Course by: Prof. Bruce Krogh.
\end{itemize}

\emph{\bf Outreach}
\\    {\bf CMU ECE Outreach} \hfill August 2013 - Apr 2014
    \begin{itemize}
    \item Host lab sessions in various ECE topics for high school kids.
    \end{itemize}



\section{Professional Activities and Service}
    {\bf Eta Kappa Nu, Carnegie Mellon University} \hfill August 2013 - Apr 2014
    \begin{itemize}
    \item ECE Honor Society.
    \end{itemize}

\end{document}